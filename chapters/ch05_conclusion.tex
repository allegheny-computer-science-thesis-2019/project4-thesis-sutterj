\chapter{Discussion and Future Work}  
\label{ch:discussionandfuturework}

The computer composer that was designed, implemented, and tested as part of this project is very different that the other composers that currently exist.  While their focus is on the composition algorithm and ability of the computer system itself, this tool attempted to bring the basic functions of a computer composition system to all kinds of users through the creation of a user interface with built in feedback mechanisms.

\vspace{\baselineskip}

While this particular tool does not feature an advanced AI powered composition, it does contain an a user interface designed to be accessible for all users.  This cannot be found on any of the other computer composers that are available.  It is in this way that this tool sets itself apart.

\vspace{\baselineskip}

There is most certainly some merit to the idea of designing a tool around the idea of accessibility.  A person might build the most advanced tool ever created, but if it never reaches the hands of the average person, can it really have the same impact?  While it is the case that not every technology is meant to shared with everyone, technologies that are used to create and make art should be available to everyone.

\vspace{\baselineskip}

Many pieces of software within the music industry are priced in such a way that they are available only to the elite and industry professionals.  This limits the possible creative output of these tools all for the sake of profit.  By making these tools accessible to more people, the potential output from these tools would be dramatically increased.

\vspace{\baselineskip}

This would lead to more music and more diverse music being developed and created.  And while this tool is not a full fledged notation program, it exists to bring music composition to people from all sorts of educational backgrounds and levels of experience.

\section{Summary of Results}
\label{sec:summaryofresults}

This section details the successes and failures of each of the major areas of the project.  This includes the development and functions of the composer and design and deployment of the user interface.

\subsection{Composer Development}
\label{subsec:summaryofresultscomposerdevelopment}

Overall, the functions that were implemented within the composer are successful and perform their desired functions.  Initially, the composer was meant to have more functionalities related to harmonization and writing parts for additional voices, but due to time constraints and developmental difficulties, the composer now only focuses on melody generation and revision.

\vspace{\baselineskip}

While it is certainly possible to implement a harmonization function, this is far more complex than the creation of a singular voiced melody.  There are many variables involved in harmonization and lots of room for user decisions and choices.  This makes development of this sort of function more challenging and makes the system of recommendations to the user much more complex.  The system is then much more difficult to use and there was simply not enough time to resolve all of theses issues.

\vspace{\baselineskip}

However, the melody generation system is very robust and contains all of the functions that it needs in order to be able to help the user generate a melody.  It may be much simpler without the harmonization system, but overall this helps the tool to better reach the overall goal of the project by making this something that anyone can use and understand.

\subsection{User Interface Development}
\label{subsec:summaryofresultsuserinterfacedevelopment}

While the user interface works correctly and provides accessible support to users of different educational backgrounds, it falls short of full integration and deployment.  What is missing in terms of integration is the connection between the analysis program, its feedback, and its display.

\vspace{\baselineskip}

Currently, the interface does not support the generation of the feedback.  To get melody feedback, the user must run the analysis program separately from the UI.  This also means that the results from the feedback are not displayed to the UI.  What this does not mean, however, is that the tool does not achieve at least part of its intended purpose.

\vspace{\baselineskip}

The biggest challenge with this area of the project is that the generation of non-musical descriptions for musical concepts is an intricate process.  It is how to explain to someone that they should only choose from a particular set of notes without having to describe to them what a key is.  The idea of choosing from a set list is not difficult, but the idea behind why this is the case is not a simple one to convey.  This same principal applies to all of the musical elements in this project.

\vspace{\baselineskip}

Additionally, what made this part of the project challenging was the intricacies of integrating external Python scripts into the user interface.  There is support for this in Django, but the process of fetching and saving unique ids as part of the database objects and the URL is extremely complicated.

\section{Future Work}
\label{sec:futurework}

This section discusses the possible extensions and revisions that could be performed on the system.  Some of these extensions were elements that were originally meant to be included in the system, but were eliminated for various reasons.  It is quite possible, however, to include them with more time and resources.

\subsection{Composer Extension}
\label{subsec:composerextension}

The most significant extensions to the composer would be the addition of functions for harmonization, additional voices, and supporting tracks.

\subsubsection{Harmonization}
\label{subsubsec:harmonization}

The idea behind this function is to generate a chord progression and then voice these chords based on the notes in the melody.  This would need to take into account voice leading and the movement between the melody and the bass note.  Adding this function would allow the user to create full fledged songs rather than just melodies.

\subsubsection{Additional Voices}
\label{subsubsec:additionalvoices}

This functions would use the chord progression generated by the harmonization function to write a complementary line to the melody.  It would add another layer to the composition and create musical interest with possible dissonance and resolution.

\subsubsection{Supporting Tracks}
\label{subsubsec:supportingtracks}

The function to create supporting tracks would generate some sort of percussion and/or bass instrument accompaniment to give the composition a more finalized sound.  This adds yet another layer of musical interest and provides a metronomic backing to the playback.

\section{Conclusion}
\label{conclusion}

In general, there were many part of this project that did not go according to plan.  The main issue that halted progress and development was related to integration with Django.  It is a very powerful tool and very helpful in the design of web interfaces, but there are many moving parts that make it difficult to extend.

\vspace{\baselineskip}

That being said, all of the individual parts and functions of the composer and the UI are developed and working.  The small bit that is missing is the integration of all of the functions into the UI and the deployment of the tool for public use.  Until the functions are fully integrated, it is not possible to deploy the tool.

\vspace{\baselineskip}

Once the tool is deployed and some additional features are added, a user study will be conducted to gain feedback on the ability of the tool to help users get through the composition process.  Based on the testing that has been done so far, it is highly likely that the tool will perform successfully.

\vspace{\baselineskip}

The areas where it may be lacking are in the guidance and explanation system, but feedback from the users will be able to indicate which parts of the system are hard to understand and use.  Once this step has been completed, the tool will undergo revisions so that it will be able to better assist in making music composition accessible for all.